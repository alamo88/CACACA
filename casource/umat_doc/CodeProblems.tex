\tophead{Things to fix in $\mu$MatIc code }


\subsection{Multiblock}
\begin{itemize}
\item Input of whole bigblock mould image
\item Perhaps unused parts need to be input so that they are available
      for the testing of neighbour states when a subblock is turned on,
      most of the  space could be saved by compression of unused subblock
      image as it will be mostly -1 and 0's

\item Checking of neighbouring blocks for mould cells upon opening

      Currently the neighbouring block is loaded by the copymat routines
      when a subblock is opened for calcualtion (switched on) However
      it is likely that the neighbouring block has not yet been switched
      on and that therefore the data arrays have not been allocated.

      This causes a segfault.
\item Virtually all data arrays added by Wei Wang for the de-centered square/octa method 
      do not have auxiliary data pointers (Value\_str ) defined in the big block.

     \paragrpaph{So...} should the data storage be redesigned to consist of a single array of data 
     structure pointers \textit{in order to} simplify fixing up the multiblock routine? Or shoudl the 
     original zoo of auxiliary data pointers be expanded to work with decentered methods?
\end{itemize}

      

\subsection{Other problems}

\begin{itemize}

\item Creation of nucleation thresholds in binary and polycomponent
      uses separate subroutines that are nearly identical. This shoudl
      be fixed to use one subroutine.

\item Mould source has instability if mould boundary intersects subblock
      wall (even for single block)

\end{itemize}



\vfill{}
\verb$Revision: 109 $
\verb$Date: 2007-11-30 12:22:27 +0000 (Fri, 30 Nov 2007) $


\verb$URL: http://xe01/svn/docs/trunk/PreparingDocs.tex $

